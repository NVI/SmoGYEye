%run:
%xelatex report.tex
%bibtex8 --wolfgang report
%xelatex report.tex

\documentclass[a4paper,12pt]{article}
\usepackage{fontspec}
\usepackage[finnish,english]{babel}

\usepackage[backend=bibtex,bibstyle=verbose,citestyle=authoryear-ibid]{biblatex}
\addbibresource{bibliography.bib}

\interfootnotelinepenalty=10000

\title{Smoke Gets in Your Eyes\\\Large{Miniproject final report}}
\author{Niko Ilomäki \and Niclas Joswig \and Sampo Savolainen}
\date{October 29, 2018}

\begin{document}

\maketitle

\section{Introduction}

The miniproject Smoke Gets in Your Eyes concentrates on wildfires and their effects on air quality. The starting point was the dataset 1.88 Million US Wildfires which is available on Kaggle. Given the effects of wildfires on air quality, air quality data was a natural pairing for the wildfire data. There is an abundance of freely available measurement data on air quality published by the U. S. Environmental Protection Agency (EPA).

There are indications that problems with wildfires are only worsening in the United States.\footnote{See for example the recent article \url{https://fivethirtyeight.com/features/wildfires-in-the-u-s-are-getting-bigger/}.} The adverse health effects of wildfire smoke have also been noted in media recently.\footnote{See e.g. \url{https://www.washingtonpost.com/news/capital-weather-gang/wp/2018/08/07/wildfire-smoke-is-wreaking-havoc-on-the-air-quality-in-the-western-u-s}}As wildfires grow in size and quantity, new solutions that help managing their impact would certainly be helpful.

Although the data used in the project pertains to the United States, a similar approach would work with other countries given similar measurement data. In Europe, countries such as Sweden and Greece have suffered greatly from wildfires recently.

\section{Target audience}

While the solution could also be repurposed for individual use (e.g., in the form of a website), the primary target audience is government agencies which could use the model to provide earlier warnings of large changes in air quality. Because the predictions come with different confidence levels and there may be other factors to consider, it seems more reasonable to start the deployment from said agencies where the predictions would be reviewed by expert staff before wider publication.

\section{Technical details}

This section explains the technical details of implementing different aspects of the project.

\subsection{Data wrangling}

Data wrangling was accomplished using Python and the \texttt{pandas} library as well as R.

The data on wildfires are available as a SQLite database. The database contains information on a total of 1.88 million fires in the U.S. between 1992 and 2015, including the discovery and containment dates of the fire, size of the fire, location of the fire, and a host of administrative information such as the agencies and units involved in containing the fire and identifiers for the fire in different systems. We discarded most of this administrative information since it was not pertinent to the task of predicting the severity of the fire. We included the information on the location of the fire (latitude and longitude coordinates), the estimated final size of the fire (in acres and as a categorical class), and the dates of the fire (from discovery to containment). This information was read into a \texttt{pandas} data frame using a SQL query.

The air quality data we used are available from the U.S. Environmental Protection Agency. We chose to use daily summaries of AQI (air quality index) divided by county, which was the most fine-grained location distinction available. The data were available as CSV files, each file containing a year of daily summaries. The summary files included location information (state and county names and codes), measurement date, AQI reading and category, the defining parameter (the most significant pollutant in the measurement), ID code for the defining site (consisting of state and county codes and a site number), and the number of sites reporting the measurement. These summaries are available from 1980 to 2018, and we downloaded the files for years corresponding to the wildfire data (1992–2015). Each CSV file was handled as its own \texttt{pandas} data frame.

The AQI summary files posed a problem: the exact location of the measurement was not included, and having merely the county information is way too coarse as a location. Fortunately, the EPA also provides a file with information the sites involved in AQI measurements, including the location, although the site ID codes needed to be parsed from separate columns containing the state, county, and site numbers. We then proceeded to add the exact coordinates for each measurement by mapping the ID code for the defining site in the AQI data with the site's coordinates.

When we had exact location data for each AQI measurement, we combined the measurements with the wildfire data. Due to time constraints, this step was done simplistically for the data used in the ML model: for each fire, we retrieved a list of AQI measurements from the time period between the discovery and containment of the fire, narrowed them down to measurements inside a 1 by 1 degree (longitude by latitude) area centered on the fire coordinates, and chose the closest measurement (if any were present). This followed from an uncertainty in establishing how far the smoke from a wildfire could impact air quality, so we chose to focus on a relatively small area around the fire.

This simplistic approach does have several limitations. First, attaching only a single AQI reading to each fire gives only a very brief glimpse at the situation: having readings from the entire duration of a fire allows more insight into the data, and we did accomplish this for some later data analysis. Second, the way the area around the fire was defined was not ideal: the area is rectangular instead of circular, which would make more sense, and since the distance covered by 1 degree of longitude depends on the latitude, the size of the search area varied between fires. However, this rough approximation was sufficient for the purposes of this project and could be improved upon in future work.

\subsection{Prediction model}

The task to predict the severity of a wildfire is displayed technically as a classification problem, which predicts the membership to the seven available classes on the basis of ten input features. The classes describe intervalls of areasizes, in which the wildfires with the fitting size are.

We tried to solve this problem with a variety of machine learning algorithms like XGBoost, Decision Trees or Linear Regression, but the best eprformance was achieved by Neural Networks.
The structure of the network can be seen in figure \cite{}.
the network consists of five blocks which contain 4 layers each. Every block starts with a Dense layer which have between 256 and 1200 Neurons. After the Dense layers follow Activation layers with the ReLu-function, which is the currently best activation-function for deep neural networks. 
To improve performance Batchnormalization layers got added into every block, to speed up learning and improve performance with data optimization. That the network overfits get prevented by Dropout layers, which have in our case a low rates from $0.2$ and $0.3$. More Dropout is not needed, because the Dataset is very big by nature, so the variety on data is huge.

After training the model with the SGDOptimizer with a learning-rate of $0.01$, a decay of $1e-6$ and a Momentum of $0.9$, we achieved 65\% accuracy as our best result.
Considering the fact that there are seven classes, so the random choice has a probability of $1/7$ to be true, the model delivers resonable results.

\subsection{Data analysis and visualization}

\section{Future directions}

It seems reasonable to assume that weather conditions, most notably the direction and speed of wind, affect where the smoke impact of a wildfire is felt. Therefore integrating them into the model could be beneficial.

In predicting how severe a wildfire can get, it could be useful to explore more data than what was possible within the scope of this project. For example, it would seem reasonable to assume the size of the fire is affected by the dryness and amount of vegetation in the area. Therefore, using weather data to determine the dryness of the region the wildfire occurs in could help with the prediction. Including the type and amount of combustible vegetation in the area could perhaps be included through explicit data, if such data is available, or roughly estimated by examining the history of wildfires in the area: if fires have occurred recently, there would be less vegetation.

Another important step would be to split the prediction into individual components of the air quality index and then reconstruct the index instead of predicting the index directly.

In terms of deploying our solution, future work could include a website that automatically updates air quality predictions for ongoing wildfires.

\section{Who did what?}

\nocite{*}
\printbibliography[title={Bibliography},heading=bibliography]

\end{document}
